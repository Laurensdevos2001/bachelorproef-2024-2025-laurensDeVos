%%=============================================================================
%% Samenvatting
%%=============================================================================

% TODO: De "abstract" of samenvatting is een kernachtige (~ 1 blz. voor een
% thesis) synthese van het document.
%
% Een goede abstract biedt een kernachtig antwoord op volgende vragen:
%
% 1. Waarover gaat de bachelorproef?
% 2. Waarom heb je er over geschreven?
% 3. Hoe heb je het onderzoek uitgevoerd?
% 4. Wat waren de resultaten? Wat blijkt uit je onderzoek?
% 5. Wat betekenen je resultaten? Wat is de relevantie voor het werkveld?
%
% Daarom bestaat een abstract uit volgende componenten:
%
% - inleiding + kaderen thema
% - probleemstelling
% - (centrale) onderzoeksvraag
% - onderzoeksdoelstelling
% - methodologie
% - resultaten (beperk tot de belangrijkste, relevant voor de onderzoeksvraag)
% - conclusies, aanbevelingen, beperkingen
%
% LET OP! Een samenvatting is GEEN voorwoord!

%%---------- Nederlandse samenvatting -----------------------------------------
%
% TODO: Als je je bachelorproef in het Engels schrijft, moet je eerst een
% Nederlandse samenvatting invoegen. Haal daarvoor onderstaande code uit
% commentaar.
% Wie zijn bachelorproef in het Nederlands schrijft, kan dit negeren, de inhoud
% wordt niet in het document ingevoegd.

\IfLanguageName{english}{%
\selectlanguage{dutch}
\chapter*{Samenvatting}
\lipsum[1-4]
\selectlanguage{english}
}{}

%%---------- Samenvatting -----------------------------------------------------
% De samenvatting in de hoofdtaal van het document

\chapter*{\IfLanguageName{dutch}{Samenvatting}{Abstract}}

Continuous Integration (CI) en Continuous Deployment (CD) zijn essentiële praktijken binnen moderne DevOps-processen. Deze bachelorproef onderzoekt en vergelijkt twee toonaangevende CI/CD-tools, Jenkins en GitHub Actions, om de meest geschikte oplossing te identificeren voor DocShifter, een bedrijf gespecialiseerd in documentconversie. Het onderzoek richt zich op vier evaluatiecriteria: prestaties, beveiliging, automatiseringsefficiëntie en integratie met bestaande systemen.\\

De methodologie omvatte een literatuurstudie en een Proof-of-Concept (PoC) waarbij beide tools werden getest op een gesimuleerde omgeving met meerdere onderling afhankelijke repositories. Jenkins bleek bijzonder krachtig in het beheren van complexe afhankelijkheden en biedt uitgebreide flexibiliteit dankzij een rijk plugin-ecosysteem. Daarentegen scoorde GitHub Actions beter op gebruiksgemak en naadloze integratie met GitHub-repositories, maar kende beperkingen bij complexe afhankelijkheden.\\

De resultaten van dit onderzoek bieden waardevolle inzichten voor organisaties die CI/CD-tools willen implementeren of optimaliseren. Hoewel beide tools sterke punten hebben, blijkt uit de evaluatie dat Jenkins beter aansluit bij de complexe en schaalbare workflows die DocShifter vereist. Deze bachelorproef draagt bij aan het onderzoeksdomein door een diepgaande vergelijking te bieden en richtlijnen te formuleren voor de selectie en implementatie van CI/CD-tools.


