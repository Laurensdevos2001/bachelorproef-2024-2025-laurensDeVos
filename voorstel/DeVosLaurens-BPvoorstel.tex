%==============================================================================
% Sjabloon onderzoeksvoorstel bachproef
%==============================================================================
% Gebaseerd op document class `hogent-article'
% zie <https://github.com/HoGentTIN/latex-hogent-article>

% Voor een voorstel in het Engels: voeg de documentclass-optie [english] toe.
% Let op: kan enkel na toestemming van de bachelorproefcoördinator!
\documentclass{hogent-article}

% Invoegen bibliografiebestand
\addbibresource{voorstel.bib}

% Informatie over de opleiding, het vak en soort opdracht
\studyprogramme{Professionele bachelor toegepaste informatica}
\course{Bachelorproef}
\assignmenttype{Onderzoeksvoorstel}
% Voor een voorstel in het Engels, haal de volgende 3 regels uit commentaar
% \studyprogramme{Bachelor of applied information technology}
% \course{Bachelor thesis}
% \assignmenttype{Research proposal}

\academicyear{2023-2024} % TODO: pas het academiejaar aan

% TODO: Werktitel
\title{Het verbeteren van compliance- en governancevereisten bij DocShifter, een gespecialiseerd bedrijf in documentconversietechnologie, door de praktische implementatie van DevSecOps, met een focus op klantgegevensbescherming en interne softwareontwikkelingsprocessen.}

% TODO: Studentnaam en emailadres invullen
\author{Laurens De Vos}
\email{laurens.devos@student.hogent.be}

% TODO: Medestudent
% Gaat het om een bachelorproef in samenwerking met een student in een andere
% opleiding? Geef dan de naam en emailadres hier
% \author{Yasmine Alaoui (naam opleiding)}
% \email{yasmine.alaoui@student.hogent.be}

% TODO: Geef de co-promotor op
\supervisor[Johnno Van De Velde]

% Binnen welke specialisatierichting uit 3TI situeert dit onderzoek zich?
% Kies uit deze lijst:
%
% - Mobile \& Enterprise development
% - AI \& Data Engineering
% - Functional \& Business Analysis
% - System \& Network Administrator
% - Mainframe Expert
% - Als het onderzoek niet past binnen een van deze domeinen specifieer je deze
%   zelf
%
\specialisation{System \& Network Administrator}
\keywords{DevSecOps, security, automation}

\begin{document}
    
    \begin{abstract}
    Dit onderzoek richt zich op het identificeren en aanpakken van concrete compliance- en governanceproblemen bij DocShifter, een bedrijf gespecialiseerd in documentconversietechnologie. Het onderzoekt de specifieke uitdagingen waarmee DocShifter wordt geconfronteerd met betrekking tot het beschermen van klantgegevens en het voldoen aan relevante regelgeving, zoals de Algemene Verordening Gegevensbescherming (AVG). Door middel van een uitgebreide literatuurstudie worden relevante aspecten van het probleemdomein geïdentificeerd, zoals de aard van de activiteiten van DocShifter, de specifieke compliance- en governancevereisten waaraan het bedrijf moet voldoen, en de potentiële impact van niet-naleving.
    
    Op basis van deze analyse wordt een methodologie ontwikkeld om de implementatie van DevSecOps als een hands-on verbetering voor het bedrijf te onderzoeken. De methodologie omvat een nulmeting om de huidige status van compliance en governance binnen DocShifter te beoordelen, gevolgd door de implementatie van DevSecOps-praktijken en -hulpmiddelen. Na de implementatie zal een meting worden uitgevoerd om de impact van DevSecOps op de compliance- en governancevereisten van DocShifter te evalueren.
    
    Door deze aanpak wordt niet alleen een dieper inzicht verkregen in de specifieke uitdagingen van DocShifter op het gebied van compliance en governance, maar wordt ook een praktische oplossing ontwikkeld die direct kan worden toegepast om de efficiëntie en effectiviteit van de bedrijfsprocessen te verbeteren. 
    
    \end{abstract}
    
    \tableofcontents
    
    % De hoofdtekst van het voorstel zit in een apart bestand, zodat het makkelijk
    % kan opgenomen worden in de bijlagen van de bachelorproef zelf.
    %---------- Inleiding ---------------------------------------------------------
    
    \section{Introductie}%
    \label{sec:introductie}
    
    In de hedendaagse digitale wereld, waar de bescherming van gegevens en naleving van regelgeving centraal staan, spelen bedrijven zoals DocShifter een cruciale rol in het leveren van veilige en conforme oplossingen. Het voorliggende onderzoek belicht de implementatie van \\ DevSecOps binnen DocShifter en verkent hoe deze aanpak kan bijdragen aan het voldoen aan de compliance- en governancevereisten van het bedrijf. Met een scherp oog voor zowel compliance als governance, zal dit onderzoek dieper ingaan op de impact van DevSecOps op verschillende facetten van het bedrijfsproces.\\
    In het kader van compliance zal dit onderzoek zich concentreren op het verbeteren van de bescherming van klantgegevens en documenten. Door een grondige analyse van bestaande processen en technologieën zal worden onderzocht hoe DevSecOps-praktijken kunnen worden geïmplementeerd om potentiële kwetsbaarheden te identificeren en te verminderen, en om te voldoen aan strikte wettelijke vereisten, zoals de Algemene Verordening Gegevensbescherming \\(AVG).\\
    Parallel aan dit streven naar compliance zal het governancegedeelte van het onderzoek zich richten op interne aspecten van de implementatie. Dit omvat het opzetten van trainingen voor ontwikkelaars in deze nieuwe werkwijze en het optimaliseren van de snelheid van softwarelevering aan klanten. Door gebruik te maken van verschillende onderzoeksmethoden zal worden onderzocht hoe governanceprincipes kunnen worden toegepast om de efficiëntie en effectiviteit van DevSecOps binnen DocShifter te verbeteren.
    Dit onderzoek belooft niet alleen een diepgaand inzicht te bieden in de rol van DevSecOps binnen DocShifter, maar ook waardevolle inzichten te verschaffen voor andere organisaties die streven naar een versterking van hun compliance- en governancekaders in een steeds complexere digitale omgeving.
    
    
    %---------- Stand van zaken ---------------------------------------------------
    
    \section{literatuurstudie}%
    \label{sec:literatuurstudie}
    \subsection{Wat is DevSecOps}
    Het concept van DevSecOps, een samentrekking van development, security en operations, vertegenwoordigt een benadering van cultuur, automatisering en platformontwerp waarbij beveiliging wordt geïntegreerd als een gedeelde verantwoordelijkheid gedurende de volledige IT-\\levenscyclus. \autocite{redhat}
    DevSecOps gaat verder dan het normale DevOps-model door niet alleen de samenwerking tussen development- en operationsteams te benadrukken, maar ook door het integreren van beveiligingsteams in het gehele ontwikkelingsproces aan te moedigen. Voorheen was beveiliging vaak beperkt tot een afzonderlijk team in de laatste fase van het project. Maar gezien de nood aan snellere ontwikkelingscycli bij DevOps, moet beveiliging een belangrijk onderdeel zijn van het gehele proces. Dit vanaf het begin van het project tot aan het einde.
    Het concept van DevSecOps onderstreept de belangrijke rol voor beveiliging vanaf het vroegste stadium van het ontwikkelingsproces. Dit toont aan dat niet alleen het automatiseren van beveiligingscontroles, maar ook het selecteren van geschikte tools en het betrekken van beveiligingsteams vanaf het begin zeer belangrijk is.
    Om effectief DevSecOps te realiseren in een project is het van cruciaal belang om beveiliging te integreren gedurende de volledige levenscyclus van applicaties. Dit houdt in dat beveiliging vroeg in het ontwikkelingsproces wordt opgenomen en dat er automatisering wordt toegepast om repetitieve taken te vereenvoudigen en de ontwikkelingssnelheid te controleren.
    DevSecOps richt zich op meer dan alleen de traditionele applicatie-ontwikkeling, maar omvat ook de opkomst van cloud technologieën zoals containers en microservices. Deze technologieën vereisen een aanpassing van beveiligingspraktijken om de nodige beveiliging van applicaties en infrastructuur te waarborgen.
    \subsection{compliance in DevSecOps.}
    Het integreren van compliance-aspecten in DevSecOps is noodzakelijk om te voldoen aan industrienormen, wetten en regelgeving. In een DevSecOps-omgeving moeten organisaties niet enkel zorgen voor veilige processen, maar ook voor naleving van regelgeving zoals GDPR (enkel voor europese landen en staat voor The General Data Protection Regulation ) \cite{Zeeshan2020}, CCPA, FISMA en andere relevante standaarden en voorschriften.
    Recente studies toonden aan dat er een groeiende interesse is in het bespreken van compliance-aspecten binnen DevSecOps, aangezien er een toename van publicaties werd vastgesteld sinds 2020. \cite{Ramaj2022} Het belang van compliance initiatie, compliance management en compliance-technische aspecten wordt steeds duidelijker. Dit omdat bedrijven geconfronteerd worden met strengere regelgeving en toenemende cybersecurity bedreigingen.
    Compliance initiatie omvat het definiëren van de gewenste staat naar de naleving van beveiligingsmaatregelen en het specificeren van de nalevingsvereisten. Compliance management omvat alle processen die verband houden met het beheer van deze naleving, zoals automatisering, testen en verificatie, validatie, controle en monitoring, bewustwording en training, en evaluatie. Compliance technische aspecten omvatten de technische middelen die gebruikt worden voor het beheren van naleving in een DevSecOps-omgeving. Met als voorbeeld het concept van compliance-as-code en verschillende nalevingstools.
    organisaties moeten investeren in geautomatiseerde nalevingscontroles, training van personeel en het gebruik van geschikte tools en technologieën om compliance effectief te waarborgen in een DevSecOps-omgeving.
    Door compliance te integreren in het ontwikkelingsproces kunnen organisaties de naleving van regelgeving verbeteren, risico's verminderen en vertrouwen opbouwen bij klanten en partners.
    \subsection{security risks in DevSecOps}
    In DevSecOps brengt de introductie van beveiliging binnenin de levenscyclus van softwareontwikkeling verschillende beveiligingsrisico's met zich mee. De complexiteit van microservices-architecturen en cloudimplementaties vergroot de kans op aanvallen. Langs de andere kant brengt de afhankelijkheid van diverse toolsets en zwakke toegangscontroles kwetsbaarheden met zich mee. Om deze uitdagingen aan te pakken, is het belangrijk voor organisaties een uitgebreid DevSecOps-model aan te nemen, waarbij er gebruik gemaakt wordt van geautomatiseerde beveiligingstests, robuuste toegangscontroles en tools voor kwetsbaarheidsbeheer. Door beveiliging als prioriteit te stellen tijdens het gehele ontwikkelingsproces, kunnen teams veerkrachtige software bouwen terwijl ze wendbaar en snel blijven. \cite{hackerone}
    \subsection{best practice voor het implementeren van DevSecOps}
    Het is belangrijk om te weten hoe we DevSecOps het best implementeren in de praktijk. Ten eerste is training essentieel. Goed ontworpen trainingsprogramma's kunnen het bewustzijn van beveiliging vergroten en individuele bijdragen aan beveiliging benadrukken. \cite{RajaviDesai2021} Ten tweede is het implementeren van beveiligingstools van uiterst belang. Deze tools moeten geïntegreerd zijn in de bestaande systemen en continue monitoring en oplossingen bieden tijdens het ontwikkelingsproces. Ten derde is het nemen van maatregelen op voorhand cruciaal. Dit omvat het beveiligen van alle componenten zoals codes, databases, netwerken. Dit door middel van verschillende controles, veilig coderen en het filteren van het verkeer.
    
    
    %---------- Methodologie ------------------------------------------------------
    \section{Methodologie}%
    \label{sec:methodologie}
    
    Om de implementatie van DevSecOps te onderzoeken en te begrijpen hoe deze kan bijdragen aan het versterken van compliance- en governancevereisten, zal er een methodologie voorgesteld worden die verschillende fasen omvat.
    Om te beginnen zal de eerste fase een grondige analyse omvatten van de huidige compliance- en governancevereisten van DocShifter. Het doel van deze eerste fase is specifieke uitdagingen en knelpunten te identificeren en duidelijke doelstellingen voor de implementatie van DevSecOps vast te stellen.
    in een volgende fase zal er een literatuurstudie worden uitgevoerd. er zal gekeken worden naar relevante literatuur over DevSecOps, compliance-frameworks zoals GDPR, en governanceprincipes. Hierbij zal speciale aandacht worden besteed aan best practices en case studies die relevant zijn voor de context van DocShifter..
    De volgende fase is een Datacollectie fase en omvat het uitvoeren van interviews, het afnemen van enquêtes en het verzamelen van relevante documentatie voor casestudies. Dit zal een alomvattend beeld opleveren van de huidige situatie en de verwachtingen ten aanzien van DevSecOps.
    De verzamelde data zal vervolgens worden geanalyseerd in de Data-analyse fase.
    Tot slot zullen de Resultaten en Conclusies worden samengevat, waarbij de verworven informatie wordt bekeken in het kader van de onderzoeksvraag en doelstellingen. Er zullen Aanbevelingen worden geformuleerd voor eventuele toekomstige implementaties van DevSecOps binnen DocShifter.

    
    %---------- Verwachte resultaten ----------------------------------------------
    \section{Verwacht resultaat, conclusie}%
    \label{sec:verwachte_resultaten}
    Er kunnen verschillende resultaten worden verwacht op basis van de voorgestelde methodologie. Allereerst wordt verwacht dat een diepgaand inzicht zal worden verkregen in de huidige compliance- en governancevereisten van DocShifter, inclusief bepaalde uitdagingen en mogelijkse knelpunten. Dit zal resulteren in een brede kennis om problemen mogelijks aan te pakken.
    Verder wordt verwacht dat er een duidelijk beeld zal ontstaan van hoe DevSecOps kan worden geïmplementeerd om deze potentiële problemen op te lossen. Hierbij zullen er gedetailleerde aanbevelingen worden gedefinieerd. Deze zullen wegwijs geven aan toekomstige stappen in de implementatie van DevSecOps binnen het bedrijf.
    Daarnaast zal verwacht worden een beter inzicht te verkrijgen uit de verworven resultaten van de interviews, enquêtes en casestudies. Deze inzichten worden gebruikt om aan te tonen hoe DevSecOps-praktijken kunnen bijdragen aan het verbeteren van de bescherming van klantgegevens en documenten, het verminderen van kwetsbaarheden en het voldoen aan wettelijke vereisten zoals GDPR.
    Verder zal een onderzoek plaatsvinden over de impact van DevSecOps op interne aspecten zoals softwareleveringssnelheid en ontwikkelaarstraining. uit dit onderzoek zullen suggesties voor verbetering voortvloeien, wat zal bijdragen aan een meer georganiseerd  en efficiënter implementatie proces.
    Tot slot zullen er in de bachelorproef meerdere aanbevelingen worden geformuleerd voor toekomstige implementaties van DevSecOps binnen DocShifter. Deze aanbevelingen zullen als een waardevolle handleiding dienen voor alle organisaties die streven naar een verbetering van hun compliance- en governancekaders.
     
    
    \printbibliography[heading=bibintoc]
    
\end{document}