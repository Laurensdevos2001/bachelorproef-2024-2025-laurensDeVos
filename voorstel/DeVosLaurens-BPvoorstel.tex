%==============================================================================
% Sjabloon onderzoeksvoorstel bachproef
%==============================================================================
% Gebaseerd op document class `hogent-article'
% zie <https://github.com/HoGentTIN/latex-hogent-article>

% Voor een voorstel in het Engels: voeg de documentclass-optie [english] toe.
% Let op: kan enkel na toestemming van de bachelorproefcoördinator!
\documentclass{hogent-article}

% Invoegen bibliografiebestand
\addbibresource{voorstel.bib}

% Informatie over de opleiding, het vak en soort opdracht
\studyprogramme{Professionele bachelor toegepaste informatica}
\course{Bachelorproef}
\assignmenttype{Onderzoeksvoorstel}
% Voor een voorstel in het Engels, haal de volgende 3 regels uit commentaar
% \studyprogramme{Bachelor of applied information technology}
% \course{Bachelor thesis}
% \assignmenttype{Research proposal}

\academicyear{2024-2025} % TODO: pas het academiejaar aan

% TODO: Werktitel
\title{Het verbeteren van compliance- en governancevereisten bij DocShifter, een gespecialiseerd bedrijf in documentconversietechnologie, door de praktische implementatie van DevSecOps, met een focus op klantgegevensbescherming en interne softwareontwikkelingsprocessen.}

% TODO: Studentnaam en emailadres invullen
\author{Laurens De Vos}
\email{laurens.devos@student.hogent.be}
\projectrepo{https://github.com/Laurensdevos2001/bachelorproef-2023-2024-laurensDeVos.git}

% TODO: Medestudent
% Gaat het om een bachelorproef in samenwerking met een student in een andere
% opleiding? Geef dan de naam en emailadres hier
% \author{Yasmine Alaoui (naam opleiding)}
% \email{yasmine.alaoui@student.hogent.be}

% TODO: Geef de co-promotor op
\supervisor{Johnno Van De Velde}

% Binnen welke specialisatierichting uit 3TI situeert dit onderzoek zich?
% Kies uit deze lijst:
%
% - Mobile \& Enterprise development
% - AI \& Data Engineering
% - Functional \& Business Analysis
% - System \& Network Administrator
% - Mainframe Expert
% - Als het onderzoek niet past binnen een van deze domeinen specifieer je deze
%   zelf
%
\specialisation{System \& Network Administrator}
\keywords{DevSecOps, security, automation}

\begin{document}
    
    \begin{abstract}
    
    DocShifter, een bedrijf gespecialiseerd in documentconversietechnologie, kampt met significante uitdagingen op het gebied van compliance en governance, vooral met betrekking tot de bescherming van klantgegevens en naleving van de Algemene Verordening Gegevensbescherming (AVG).
    
    Om deze uitdagingen aan te pakken, is een uitgebreide literatuurstudie uitgevoerd. Deze studie analyseert de specifieke activiteiten van DocShifter, de compliance- en governancevereisten waaraan het moet voldoen, en de mogelijke gevolgen van niet-naleving. Op basis van deze analyse is een methodologie ontwikkeld, waarin een nulmeting de huidige status van compliance en governance evalueert. Vervolgens worden DevSecOps-praktijken geïmplementeerd, waarna de impact op de naleving van regelgeving en gegevensbescherming wordt gemeten.
    
    De resultaten van dit onderzoek bieden waardevolle inzichten in de compliance- en governance-uitdagingen van DocShifter. Bovendien wordt een praktische oplossing voorgesteld die de efficiëntie en effectiviteit van de bedrijfsprocessen verbetert, wat bijdraagt aan een versterkte naleving van regelgeving en betere gegevensbescherming.
    
    
    \end{abstract}
    
    \tableofcontents
    
    % De hoofdtekst van het voorstel zit in een apart bestand, zodat het makkelijk
    % kan opgenomen worden in de bijlagen van de bachelorproef zelf.
    %---------- Inleiding ---------------------------------------------------------
    
    \section{Introductie}%
    \label{sec:introductie}

    In de hedendaagse digitale wereld is het voor bedrijven cruciaal om te voldoen aan strikte compliance- en governancevereisten, met name op het gebied van klantgegevensbescherming en interne softwareontwikkelingsprocessen. DocShifter, een vooraanstaande speler in documentconversietechnologie, wordt momenteel geconfronteerd met specifieke uitdagingen bij het naleven van de Algemene Verordening Gegevensbescherming (AVG). Deze uitdagingen hebben betrekking op het ontoereikend versleutelen van klantgegevens of het niet tijdig en snel genoeg updaten van software om aan de vereisten te voldoen, wat de organisatie blootstelt aan aanzienlijke risico's op legaal vlak.
    
    \noindent De huidige aanpak van DocShifter blijkt ontoereikend te zijn, omdat veiligheidsmaatregelen over het hoofd kunnen gezien worden door een minder goede aanpak van software uit te brengen. Dit roept de vraag op hoe DocShifter zijn compliance- en governancekaders kan versterken om beter te voldoen aan de AVG en andere relevante normen.
    
    \noindent De centrale onderzoeksvraag in dit onderzoek luidt dan ook: Hoe kan de implementatie van DevSecOps bijdragen aan het verbeteren van de\\ compliance- en governancevereisten van DocShifter, specifiek in het kader van de AVG? Deze vraag wordt opgesplitst in de volgende deelvragen:
    
    \begin{itemize}
        \item  Wat zijn de huidige tekortkomingen in de compliance- en governanceaanpak van DocShifter ten aanzien van de AVG? Welke risico's lopen zij door deze tekortkomingen?
    \end{itemize}
    
    \begin{itemize}
        \item  Hoe kunnen DevSecOps-praktijken specifiek bijdragen aan het verbeteren van de naleving van de AVG bij DocShifter? Welke methoden en tools zijn het meest geschikt voor deze implementatie?
    \end{itemize}
    
    \noindent Op basis van deze deelvragen worden de volgende doelstellingen geformuleerd:
    
    \begin{itemize}
        \item Het in kaart brengen van de huidige \\ compliance- en governance-uitdagingen bij DocShifter met betrekking tot de AVG.
    \end{itemize}
    
    \begin{itemize}
        \item Het ontwikkelen en voorstellen van een \\ DevSecOps-strategie die specifiek gericht is op het aanpakken van deze uitdagingen.
    \end{itemize}
    \begin{itemize}
        \item Het evalueren van de effectiviteit van deze strategie in termen van verbeterde compliance en gegevensbescherming.
    \end{itemize}
    
    \noindent Door een grondige analyse van deze uitdagingen en mogelijke oplossingen, streeft dit onderzoek ernaar om concrete aanbevelingen te doen die de efficiëntie en effectiviteit van de bedrijfsprocessen bij DocShifter zullen verbeteren.
    
    
    
    
    %---------- Stand van zaken ---------------------------------------------------
    
   \section{Literatuurstudie}%
   \label{sec:literatuurstudie}
   \subsection{Inleiding tot DevSecOps en Compliance}
   DevSecOps, een samenvoeging van development, security en operations, is een benadering die beveiliging integreert als een gedeelde verantwoordelijkheid gedurende de volledige \\ IT-levenscyclus. Volgens Red Hat (2020) gaat dit concept verder dan het traditionele DevOps-model door beveiliging te betrekken vanaf de vroegste stadia van ontwikkeling en doorlopend tot de eindfase van het project.
   Traditioneel werd beveiliging vaak gezien als een afzonderlijke taak die pas aan het einde van het ontwikkelingsproces werd uitgevoerd. Met de toegenomen behoefte aan snellere ontwikkelingscycli in DevOps, is het echter cruciaal geworden om beveiliging continu en integraal te maken in het gehele proces. Dit betekent dat beveiligingsteam vanaf het begin betrokken zijn bij het project, en dat beveiligingscontroles worden geautomatiseerd en geschikte tools worden geselecteerd.
   De nadruk van DevSecOps ligt op het belang van vroegtijdige en doorlopende beveiliging. Dit benadrukt niet alleen het belang van het automatiseren van beveiligingscontroles, maar ook de noodzaak om beveiligingsteams vanaf het begin van het ontwikkelingsproces te betrekken.
   Effectieve implementatie van DevSecOps vereist dat beveiliging wordt geïntegreerd gedurende de volledige levenscyclus van applicaties. Automatisering speelt een cruciale rol in het vereenvoudigen van repetitieve taken en het handhaven van de ontwikkelingssnelheid. Bovendien breidt DevSecOps zich uit naar moderne technologieën zoals containers en microservices, die aangepaste beveiligingspraktijken vereisen om \\ applicatie- en infrastructuurbeveiliging te waarborgen \autocite{redhat2023}.
   
   \subsection{Belang van governance en compliance in DevSecOps}
   
  In DevSecOps is governance van cruciaal belang om ervoor te zorgen dat beveiliging en compliance geïntegreerd worden binnen de ontwikkelings- en operationele processen. Dit gaat verder dan alleen het toevoegen van beveiligingscontroles; het vereist een gestructureerde benadering waarbij beleidsregels, standaarden en compliance-eisen consistent worden nageleefd en toegepast in de volledige softwarelevenscyclus.
  
  Een Gartner-enquête uit 2022 toonde aan dat 90\% van de organisaties de noodzaak ziet om governance en compliance te verbeteren door een betere integratie van beveiligingsoperaties \autocite{GlobalSign2022}. DevSecOps maakt dit mogelijk door beveiligingscontroles vroeg in de softwareontwikkelingscyclus te integreren, waardoor risico's beter beheerst kunnen worden en compliance automatisch kan worden gehandhaafd. Traditioneel werden beveiliging en compliance als losse stappen aan het einde van een ontwikkelproces toegevoegd, wat vaak leidde tot vertragingen en hoge kosten. Met DevSecOps worden beveiliging en compliance echter onderdeel van het ontwikkelproces, waardoor problemen vroegtijdig kunnen worden opgespoord en opgelost.
  
  Volgens de OWASP DevSecOps-richtlijnen is het hanteren van een "shift-left" benadering een sleutelelement van governance binnen DevSecOps. Deze benadering houdt in dat beveiligings- en complianceactiviteiten al vroeg in het ontwikkelproces worden geïmplementeerd, bijvoorbeeld tijdens het coderen en testen. Dit verkleint niet alleen de kans op beveiligingsincidenten, maar zorgt ook voor een continue naleving van regelgeving gedurende de hele levenscyclus van de applicatie.\autocite{Project2019}
  
  Kortom, goede governance in DevSecOps stelt organisaties in staat om de naleving van regelgeving en beveiligingsstandaarden effectief te waarborgen. Dit verbetert niet alleen de veiligheid van softwaretoepassingen, maar helpt ook bij het voldoen aan wettelijke en industriële normen zoals ISO 27001 en AVG, wat essentieel is voor organisaties die opereren in gereguleerde sectoren.
  
   
   \subsection{Automatisering van compliance binnen DevSecOps}
   
   Automatisering speelt een cruciale rol in het handhaven van compliance binnen DevSecOps volgens Odorfer (2021). Door beleid als code (policy as code) te implementeren, kunnen organisaties de naleving van beveiligingsnormen en -procedures op een consistente en schaalbare manier waarborgen. Volgens de AWS Security Blog stelt beleid als code teams in staat om beveiligings- en compliance-eisen te automatiseren door regels in code te definiëren, die vervolgens automatisch kunnen worden toegepast en gecontroleerd in de ontwikkelings- en operationele processen.
   
   Deze aanpak minimaliseert niet alleen menselijke fouten, maar zorgt ook voor een snellere feedbacklus, waardoor teams proactief kunnen reageren op compliance-issues voordat ze problematisch worden. Door tools en technologieën in te zetten die beleidsregels automatisch toepassen, kunnen organisaties de risico's van niet-naleving verlagen en de snelheid van softwarelevering verhogen, wat cruciaal is in de huidige snel veranderende digitale omgeving.
   
   Bovendien maakt de automatisering van compliance het mogelijk om voortdurend te monitoren en rapporteren over de naleving, waardoor teams beter in staat zijn om te voldoen aan zowel interne beleidslijnen als externe regelgeving, zoals de AVG en andere industriestandaarden.\autocite{Odorfer2021}
   
   \subsection{Specifieke regelgeving en standaarden relevant voor DevSecOps}
   
  In het kader van DevSecOps is het essentieel om op de hoogte te zijn van specifieke regelgeving en standaarden die invloed hebben op de ontwikkeling en het beheer van softwareapplicaties. Een belangrijke wetgeving in Europa is de Algemene Verordening Gegevensbescherming (AVG), die strikte richtlijnen biedt voor de bescherming van persoonlijke gegevens. De AVG verplicht organisaties om passende technische en organisatorische maatregelen te nemen om de privacy van klanten te waarborgen en hen te beschermen tegen gegevensinbreuken .
  
  De AVG legt ook de verantwoordelijkheid bij organisaties om transparant te zijn over hoe ze gegevens verzamelen, opslaan en verwerken. Dit is cruciaal voor DevSecOps-teams, aangezien zij beveiliging en compliance in de vroege stadia van de softwarelevenscyclus moeten integreren. Door de richtlijnen van de AVG te volgen, kunnen organisaties niet alleen hun juridische verplichtingen nakomen, maar ook het vertrouwen van klanten vergroten door transparante en veilige gegevensverwerking.
  
  Bovendien kunnen DevSecOps-teams profiteren van best practices en richtlijnen van andere organisaties, zoals de Open Web Application Security Project (OWASP), die helpen bij het implementeren van veilige ontwikkelingspraktijken en het handhaven van compliance met relevante regelgeving.
  \autocite{EU2022}
  
  \subsection{Best practices voor compliance-gedreven DevSecOps-implementatie}
  
  Volgens OWASP (2021) is het implementeren van best practices voor compliance-gedreven DevSecOps essentieel voor het effectief integreren van beveiliging en compliance binnen de softwareontwikkelingslevenscyclus. Volgens de OWASP DevSecOps Guideline moeten organisaties een proactieve aanpak hanteren die gericht is op het integreren van beveiligingsmaatregelen in alle fasen van de ontwikkeling. Dit betekent dat beveiliging niet als een afzonderlijke stap aan het einde van het ontwikkelingsproces moet worden gezien, maar als een continu proces dat begint met de planningsfase en zich uitstrekt tot en met de implementatie en het onderhoud van de software.\autocite{OWASP2021}
  
  Een belangrijk aspect van deze proactieve benadering is het automatiseren van beveiligingscontroles en compliance-eisen. Puppet benadrukt het belang van automatisering in hun blog over een proactieve beveiligingsaanpak, waarbij ze aangeven dat automatisering helpt bij het minimaliseren van menselijke fouten en het versnellen van het compliance-proces (Puppet, 2022). Dit maakt het mogelijk om consistentie en betrouwbaarheid in beveiligingsmaatregelen te waarborgen en snel te reageren op eventuele compliance-issues die zich kunnen voordoen.\autocite{Puppet2022}
  
  Red Hat wijst ook op het belang van een integrale DevSecOps-aanpak, waarbij samenwerking tussen ontwikkeling, beveiliging en operaties wordt bevorderd. Dit bevordert een cultuur van gezamenlijke verantwoordelijkheid voor beveiliging en compliance, wat essentieel is voor het succes van een DevSecOps-implementatie (Red Hat, 2023). Door teams in staat te stellen om samen te werken en gedeelde doelen te hebben, kunnen organisaties beter inspelen op de voortdurende veranderingen in regelgeving en beveiligingseisen.\autocite{RedHat2021}
  
  
  
  
  
    
    %---------- Methodologie ------------------------------------------------------
    \section{Methodologie}%
    \label{sec:methodologie}
    
    \noindent Om de implementatie van DevSecOps te onderzoeken met een focus op governance- en compliancevereisten, wordt de methodologie voorgesteld die uit drie fasen bestaat. 
    
    \subsection{Fase 1: Analyse van huidige compliance- en governancevereisten}
    \noindent De eerste fase omvat een grondige analyse van de huidige compliance- en governancevereisten van DocShifter. Dit proces zal inhouden:
    
    \begin{itemize}
        \item Stakeholder interviews: Het uitvoeren van interviews met een select aantal relevante stakeholders, zoals ontwikkelaars en DevOps-engineers, om hun ervaringen en percepties van de huidige compliance-eisen te verzamelen.
        \item Procesanalyse: Gelijktijdig met de interviews worden de bestaande processen en workflows binnen DocShifter in kaart gebracht om inefficiënties en risico’s te identificeren.
    \end{itemize}
    
    \subsection{Fase 2: Literatuurstudie}
    \noindent Een literatuurstudie wordt uitgevoerd die specifiek ingaat op de governance- en compliancevereisten, met nadruk op:
    
    \begin{itemize}
        \item Best Practices en Case Studies: Het identificeren van best practices en relevante case studies die specifiek gericht zijn op de geïdentificeerde vereisten. Deze inzichten zullen helpen bij het vormen van een solide basis voor de implementatie van DevSecOps.
    \end{itemize}
    
    \subsection{Fase 3: Evaluatie en Aanbevelingen}
    \noindent Op basis van de inzichten uit de literatuurstudie en de eerdere analyse worden aanbevelingen gedaan voor de implementatie van DevSecOps binnen DocShifter. Dit omvat:
    
    \begin{itemize}
        \item Longlist en Shortlist van Oplossingen: Het opstellen van een shortlist van tools en frameworks die kunnen bijdragen aan de governance en compliance, inclusief een evaluatie op effectiviteit en toepasbaarheid.
        \item Proof of Concept: Het uitvoeren van een beperkte implementatie van de geselecteerde oplossingen om de haalbaarheid en effectiviteit in de praktijk te testen.
    \end{itemize}
    
    %---------- Verwachte resultaten ----------------------------------------------
    \section{Verwacht resultaat, conclusie}%
    \label{sec:verwachte_resultaten}
    Het onderzoek zal naar verwachting resulteren in een diepgaand inzicht in de huidige compliance- en governance-uitdagingen van DocShifter. Door middel van een systematische analyse en literatuurstudie wordt verwacht dat er een duidelijke strategie zal worden ontwikkeld voor de implementatie van DevSecOps binnen het bedrijf. Deze strategie resulteert in een selectie van relevante tools en best practices die specifiek zijn afgestemd op de behoeften van DocShifter. Bovendien zal de Proof of Concept een praktisch bewijs leveren van de effectiviteit van deze aanpak, wat zal bijdragen aan het verbeteren van interne processen.
    
    In conclusie zal dit onderzoek waardevolle inzichten bieden in hoe DocShifter zijn compliance- en governancevereisten kan versterken door de implementatie van DevSecOps. De voorgestelde oplossingen zullen niet alleen de huidige uitdagingen aanpakken, maar ook een duurzame basis leggen voor toekomstige verbeteringen in de beveiliging en efficiëntie binnnen DocShifter. De resultaten van dit onderzoek zullen DocShifter in staat stellen om beter te voldoen aan relevante regelgeving, wat uiteindelijk zal bijdragen aan een versterkt vertrouwen van klanten en stakeholders.
    
    
         
    
    \printbibliography[heading=bibintoc]
    
\end{document}