\documentclass[a0,portrait]{hogent-poster}

% Info over de opleiding
\course{Bachelorproef}
\studyprogramme{Toegepaste Informatica}
\academicyear{2024-2025}
\institution{Hogeschool Gent, Valentin Vaerwyckweg 1, 9000 Gent}

% Info over de bachelorproef
\title{Optimalisatie van DevOps-processen met CI/CD-tools: Vergelijking van Jenkins en GitHub Actions}
\subtitle{}
\author{Laurens De Vos}
\email{laurens.devos@student.hogent.be}
\supervisor{Dhr. A. Van Maele}
\cosupervisor{Dhr. J. Van De Velde (DocShifter)}
\specialisation{Systeem- en Netwerkbeheer}
\keywords{CI/CD, DevOps, Jenkins, GitHub Actions}
\projectrepo{https://github.com/BPLaurensDeVos}

\begin{document}
    
    \maketitle
    
    \begin{abstract}
        Continuous Integration (CI) en Continuous Deployment (CD) zijn essentiële processen binnen DevOps. Deze bachelorproef onderzoekt en vergelijkt Jenkins en GitHub Actions, met als doel de meest geschikte oplossing te identificeren voor het optimaliseren van de softwareontwikkeling bij DocShifter. Door middel van literatuuronderzoek en een Proof-of-Concept worden prestaties, beveiliging, schaalbaarheid en automatiseringsefficiëntie beoordeeld. De resultaten bieden praktische aanbevelingen voor de implementatie van CI/CD-tools.
    \end{abstract}
    
    \begin{multicols}{2}
        
        \section*{Introductie}
        DevOps streeft naar het versnellen van softwareontwikkeling en het verbeteren van samenwerking tussen teams. CI/CD speelt hierin een sleutelrol door code sneller en betrouwbaarder te integreren en implementeren. Voor DocShifter, een bedrijf dat zich richt op documentconversie, is het essentieel om efficiënte en veilige workflows te hebben. De vraag luidt:
        \begin{quote}
            \textit{Welke CI/CD-tool is het meest geschikt om de DevOps-processen van DocShifter te optimaliseren?}
        \end{quote}
        
        \section*{Achtergrond}
        \textbf{Jenkins:}
        \begin{itemize}
            \item Open-source CI/CD-tool met een uitgebreid plugin-ecosysteem.
            \item Zeer flexibel, maar vereist veel configuratie.
            \item Geschikt voor complexe workflows en grote teams.
        \end{itemize}
        \textbf{GitHub Actions:}
        \begin{itemize}
            \item Geïntegreerd met het GitHub-ecosysteem.
            \item Gebruiksvriendelijk en eenvoudig te configureren via YAML-bestanden.
            \item Beperkt voor complexe afhankelijkheden.
        \end{itemize}
        
        \section*{Doelstellingen}
        
        \begin{enumerate}
            \item Vergelijken van Jenkins en GitHub Actions op prestaties, beveiliging, schaalbaarheid en integratie.
            \item Identificeren van sterktes en zwaktes van beide tools.
            \item Formuleren van aanbevelingen voor DocShifter.
        \end{enumerate}

        \section*{Methodologie}
        \begin{itemize}
            \item Literatuurstudie naar CI/CD-praktijken en tools.
            \item Proof-of-Concept (PoC): implementatie en evaluatie van workflows in beide tools.
            \item Meetbare evaluatiecriteria:
            \begin{itemize}
                \item Buildtijden
                \item Foutpercentages
                \item Schaalbaarheid
                \item Integratiegemak
            \end{itemize}
        \end{itemize}
        
        \section*{Proof-of-Concept (PoC)}
        Om de verschillen tussen Jenkins en GitHub Actions beter te begrijpen, is een Proof-of-Concept (PoC) uitgevoerd. Deze PoC omvatte het implementeren van CI/CD-workflows voor meerdere repositories met onderlinge maven afhankelijkheden. Tijdens de implementatie werden beide tools beoordeeld op een reeks meetbare criteria:
        
        \subsection*{Uitvoering van de PoC}
        \begin{itemize}
            \item Workflows werden ontwikkeld voor het bouwen, testen en implementeren van softwarecomponenten.
            \item Zowel eenvoudige als complexe scenario’s werden gesimuleerd om schaalbaarheid en flexibiliteit te testen.
            \item Automatiseringsscripts en configuratiebestanden (bijvoorbeeld YAML voor GitHub Actions) werden geoptimaliseerd voor gebruiksgemak en consistentie.
        \end{itemize}
        
        \subsection*{Evaluatiecriteria}
        De tools werden vergeleken op de volgende punten:
        \begin{itemize}
            \item \textbf{Buildtijd}: De snelheid waarmee software werd gebouwd en getest.
            \item \textbf{Betrouwbaarheid}: Het foutpercentage tijdens builds en deployments.
            \item \textbf{Schaalbaarheid}: De mogelijkheid om meerdere taken parallel te verwerken.
            \item \textbf{Integratiegemak}: Hoe eenvoudig beide tools te integreren zijn met bestaande systemen en tools zoals Git en java (Maven).
        \end{itemize}
        
        \subsection*{Belangrijkste Resultaten}
        \begin{itemize}
            \item GitHub Actions bleek beter geschikt voor kleinere en minder complexe workflows, dankzij de naadloze integratie met het GitHub-ecosysteem en eenvoudige configuratie via YAML-bestanden.
            \item Jenkins blonk uit in het omgaan met complexe, grootschalige workflows, mede door het uitgebreide plugin-ecosysteem en hoe het omging met afhankelijkheden tussen repositories.
            \item Bij eenvoudige projecten waren de buildtijden van GitHub Actions korter, maar Jenkins toonde meer consistentie bij grotere workflows.
        \end{itemize}
        
        \section*{Conclusies}
        \begin{itemize}
            \item Jenkins wordt aanbevolen voor complexe en schaalbare workflows.
            \item GitHub Actions is ideaal voor eenvoudigere projecten binnen het GitHub-ecosysteem, vooral zonder afhankelijkheden tussen repositories.
            \item Beide tools vereisen robuuste beveiligingsmaatregelen.
        \end{itemize}
        
        \section*{Aanbevelingen}
        \begin{itemize}
            \item Gebruik Jenkins voor geavanceerde automatisering en grootschalige projecten.
            \item Gebruik GitHub Actions voor snellere en minder complexe implementaties.
            \item Implementeer sterke geheimbeheerprocedures in beide tools.
        \end{itemize}
        
        \section*{Toekomstig Onderzoek}
        \begin{itemize}
            \item Integratie van DevSecOps in CI/CD-tools.
            \item Vergelijkend onderzoek tussen Jenkins en een andere CI/CD-tool zoals Azure Devops of spacelift.
            \item Onderzoek naar hybride oplossingen die beide tools combineren.
        \end{itemize}
        
    \end{multicols}
    
\end{document}
